\documentclass[10pt,pdf,hyperref={unicode}]{beamer}
\usepackage[utf8]{inputenc}
\usepackage[T1,T2A]{fontenc}
\usepackage[english, russian]{babel}
\usepackage{graphicx}
\usetheme{Rochester}

\title{\huge Практикум. Задание 4 }
\author{Чернобай Анна}
\date{Москва 2022}

\begin{document}
\frame[plain]{\titlepage}

\section{Анализ исходных данных}

\begin{frame}
\frametitle{Количество произведенных и сломавшихся мечей для каждой из компании-поставщика за все время}
\begin{figure}[t]
    \centering
    \includegraphics[width=1.0\textwidth]{all_time.png}
\end{figure}
Из построенных графиков видим, что за все время работы количество произведенного оружия для обеих компаний практически одинаковое. Однако число сломанных мечей из стали компании Westeros Inс. заметно больше, чем у Harpy \& Co.
\end{frame}

\begin{frame}
\frametitle{Количество произведенных и сломавшихся мечей для каждой компании-поставщика в каждый месяц}
\begin{figure}[t]
    \centering
    \includegraphics[width=1.0\textwidth]{monthly.png}
\end{figure}
Из построенных графиков видим, что число произведенного оружия за каждый месяц для обеих компаний приблизительно равно. Также обратим внимание, что со временем качество стали Harpy \& Co и Westeros Inс. улучшилось, однако у Harpy \& Co это улучшение значительнее.
\end{frame}

\begin{frame}
\frametitle{Соотношение числа сломавшихся мечей и общего числа выкованных за каждый месяц}
\begin{figure}[t]
    \centering
    \includegraphics[width=0.8\textwidth]{monthly_percents.png}
\end{figure}
Из построенных графиков видим, что к последнему месяцу процент сломанных изделий снижается у обеих компаний. В целом, все также, качество продукции из стали Harpy \& Co, кажется, более надежным.
\end{frame}

\begin{frame}
\frametitle{Количество поломок в зависимости от кузнеца для каждой компании-поставщика}
\begin{figure}[t]
    \centering
    \includegraphics[width=0.8\textwidth]{unsullen.png}
\end{figure}
Из построенных графиков видим, что в целом, количество поломанных мечей не сильно зависит от кузнеца для обеих компаний-поставщиков. Также можно заметить, что доля поломанных мечей больше для мечей из стали Westeros Inс.
\end{frame}


\begin{frame}
\frametitle{Количество поломанной продукции после каждого месяца экспуатации для каждой компании поставщика}
\begin{figure}[t]
    \centering
    \includegraphics[width=0.8\textwidth]{how_long_until_broken.png}
\end{figure}
Из построенных графиков видим, что количество мечей из стали Westeros Inс., сломавшихся уже после первого месяца эксплуатации, больше чем для Harpy \& Co. Однако с четвертого месяца эксплуатации мечи из стали Harpy \& Co, ломаются чаще (но к 4 месяцу их и осталось больше, т.к. до этого они ломались меньше)
\end{frame}

\section{Итог}

\begin{frame}
\frametitle{\insertsection}
 Поломка мечей зависит по большей мере от качества стали.
  
    
    \begin{block}{Вывод}
    Исходя из вышенаписанного, договор, на мой взгляд, следует заключить с компанией Harpy \& Co. Сталь этой компании обладает лучшим качеством в сравнении с Westeros Inс.
    \end{block}
\end{frame}

\end{document}
